%%
%% This is file `example/app_lhospital.tex',
%% generated with the docstrip utility.
%%
%% The original source files were:
%%
%% install/buptgraduatethesis.dtx  (with options: `app-lhospital')
%% 
%% This file is a part of the example of BUPTGraduateThesis.
%% 

\chapter{不定型($0/0$)极限的计算}
\begin{theorem}[L'Hospital法则]
  若
  \begin{enumerate}
  \item 当 $x \to a$ 时,函数 $f(x)$ 和 $g(x)$ 都趋于零;
  \item 在点 $a$ 某去心邻域内,$f'(x)$ 和 $g'(x)$ 都存在,且 $g'(x)\neq 0$;
  \item $\displaystyle\lim_{x \to a} \dfrac{f'(x)}{g'(x)}$ 存在(或为无穷大),
  \end{enumerate}
  那么
  \begin{align}
    \label{eq:app:lhospital}
    \lim_{x \to a} \frac{f(x)}{g(x)} = \lim_{x \to a} \frac{f'(x)}{g'(x)}.
  \end{align}
\end{theorem}
\begin{proof}
  以下只证明两函数 $f(x)$ 和 $g(x)$ 在 $x = a$ 为光滑函数的情形。
  由于 $f(a) = g(a) = 0$,原极限可以重写为
  \begin{align*}
    \lim_{x \to a} \frac{f(x) - f(a)}{g(x) - g(a)}.
  \end{align*}
  对分子分母同时除以 $(x - a)$,得到
  \begin{align*}
    \lim_{x \to a} \frac{%
      \dfrac{f(x) - f(a)}{x - a}
    }{%
      \dfrac{g(x) - g(a)}{x - a}
    } &
    = \frac{%
      \displaystyle\lim_{x \to a} \frac{f(x) - f(a)}{x - a}
    }{%
      \displaystyle\lim_{x \to a} \frac{g(x) - g(a)}{x - a}
    }.
  \end{align*}
  分子分母各得一差商极限,即函数 $f(x)$ 和 $g(x)$ 分别在 $x = a$ 处的导数
  \begin{align*}
    \lim_{x \to a} \frac{f(x)}{g(x)} &
    = \frac{f'(a)}{g'(a)}.
  \end{align*}
  由光滑函数的导函数必为一光滑函数,故 \eqref{eq:app:lhospital} 得证。
\end{proof}
